\chapterwithauthor{Alexandra Leisse}{那些我很庆幸曾经不知道的事}

\authorbio{Alexandra Leisse离开了一个舞台而来到了另一个,并把她对软件和网络的热情变成了职业。
在经历了十二个月的为软件和歌剧自由撰稿的过渡期,以及在KDE相关活动投入不计其数的时间后,
她加入了诺基亚的Qt开发框架团队,成为了一位社区管理员。

\newline
她是那位存在于Qt开发者网络和在线Qt社区活动幕后的女人。
尽管在歌剧表演上颇有成就,她几乎拒绝在公开场合歌唱。}

\section*{引言}

当Lydia邀请我加入她的题为“那些我希望当时就知道的事”的新书项目时,我的思绪一片空白。
那些我希望当时就知道的却不知道的事?我想不起来有这样的事。

我不是在说我不需要学习任何事,相反,我必需学习很多并且在过程中犯下了数不清的错误。
但是那其中有我期望避免的情况或错误吗?我认为没有。

我们所有人都有这样的烦人的趋向,就是在意那些我们可以做得更好和我们所不知的事,
并且认为这些是弱点。但当弱点成为我们的强项时又会如何?

以下是我个人的关于无知、天真和错觉的故事,也是我自己也不知道我有多么快乐的故事。

\section*{名字}

当时我并不知道那个在我工作第一天遭遇的那位同事是谁。
他进入了房间,自我介绍后开始问我各种难题,这给了一种那些我将要做的事情都是没有意义的印象。
他显然很清楚我将要在KDE社区从事的工作以及我之前打交道的人。
然而我们看起来有不同的立场。在一些观点上,我逐渐对他的挑衅感到厌烦并且失去耐心。
我告诉他和人打交道这类事情并不总是像工程师们认为的那样容易。

我们大约进行了一个小时的讨论,直到他离开后,我才谷歌了一下他的名字:Matthias Ettrich。
我阅读了他的事迹,从而明白他为何会问那些问题。如果我早点知道他是KDE项目的建立者之一,
那么我很可能会以另一种方式和他争论————当然这只是假设。

在过去一年里,我必须查找相当数量的名字,不过我很高兴每次都是在初次联系\textit{后}再去查找。

这也许是我最重要的观点。当我第一次见到那些FOSS的人时,我几乎从未听说过他们的名字。
我不知道他们的背景,他们的长处,或者是他们的失败事迹。我以同样的方式对待每一个人:平等。

因为无知(或者像某些人所说的,天真),在开始我的FOSS旅程后,对于那些遇见的人,我并没有感到低人一等。
我知道我需要学习很多东西,但我从不觉得,作为一个人而言,我处于一个比他人低等的位置。

\section*{“高姿态的项目”}

我并没有认真地去追随dot.kde.org或者是PlanetKDE,当我在KDE的邮件列表潜水前,
不会去关注那些数不清的FOSS相关的发表内容。我认为那些渠道首先并且重点都是针对特定读者的,
主要是这个项目的用户和贡献者。

有相当一段时间,我从未想过那些我发表在Dot上的文章会被记者采纳。
我花精力来写文章是因为我想做好自己的工作而不是害怕他人觉得我是个傻瓜。
出版文章的选择是其他人负责的,而且我写的东西对我自己来说也并不重要。
我曾经想和某些人交流,而官方的渠道和我的个人博客看来是最有效的方式。

当我在自己博客上宣布我要开始一项新工作后,这居然会在ReadWriteWeb上被引用,使我大吃一惊。
并不是因为我不知道他们读了我写的文章——我当然希望他们有读过,我只是不想这事变得太话题而已。
这连夏日休假都算不上。

没人告诉我好事,我宁愿没发表过一行字。

\section*{局外人}

记得之前我首次参加会议时,秉持着我与在场其他人都不同的信念。
我认为我是局外人,除了对技术有着浓厚兴趣外,我和其他人没有多少共同点:
在大学毕业后几年里我成为了自由工作者,我也没有受过这领域的相关教育,
而且我已经是一个十岁小孩的母亲。
至少在纸面上,就这和一个普通人接触FOSS项目没什么区别。

在2008年,作为KDE市场和宣传团队的一员,我参加了一个KOffice的冲刺会议,
为2.0版本的发布而作准备。最初的想法是为发布而策划一系列宣传活动来增加开发者和用户。
为此我们中的三人同时和开发者们讨论着一些相关事项。

我们尝试着去弄清楚如何将KOffice定位,并且针对听众而改变沟通的策略。
在这过程中,我们很快发现不得不退后一步:在那个时间点,这个办公套件的不成熟使得
它不可能成为可信任用户的选择。我们着眼点不得不停留在开发者和早期用户。
对一些开发者,这是真是一项累人的工作,但是作为局外人我们有机会更平常地看待这个软件,
不用去回忆那些已经融进代码的心血、甜蜜和泪水。

对于很多项目来说,无论是哪种类型,核心贡献者都很难采取客观的态度来看待一些事情。
我们尝试不去关注完成的整体方面,而是非常注重一些细节问题,或者其他相关的地方。
有时候我们会错过一个好机会,因为我们\textit{认为}它和我们所做的事情无关——
或者没人会把它作为优先事项。

在所有的这些情况中,项目外的人在讨论中很可能会提出不同的见解,
特别是在决定优先事项的时候。如果他们自己不是开发者就更有帮助了:
他们会问不同的问题,却不会因为知道和理解所有技术细节感到压力,
有了他们的帮助,决定和交流的层次得以提升。

\section*{结论}

忽视是幸运的。这不仅是对那些因为不知道而不会担心从而受益的个人来说,
对这些人参与的项目也是。他们带来了不同的观点和经历。

所以,去发现你感兴趣的项目,不要计较你知道的有多少。
