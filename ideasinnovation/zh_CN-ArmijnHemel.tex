\chapterwithauthor{Armijn Hemel}{写了再说}

\authorbio{自从1994年,Armijn Hemel的哥哥带回家一堆装有FreeBSD早期版本的软盘,Armijn Hemel
就开始使用自由软件。一年后,他决定使用linux,并且自此以后,不论是在家里,Utrecht大学还是在工作上,
他都一直使用类UNIX操作系统。
从2005年起,Armijna成了gpl-voilations.org组织的核心团队成员,同时让他自己的咨询公司(Tjaldur Software Governance Solutions)
专注与GPL协议违规的检测和解决。
}

\noindent{}追溯到1999年,我开始参与FLOSS的行动。那时,我使用Linux和FreeBSD已经有些年份了,但是仅仅作为
使用者。我想做些实际的贡献,但是找不到合适的项目参与进去,所以我决定开始我自己的项目。回想起开始的原因,我想有很多。
一个因素是,我不确定自己的代码是否够优秀,以至于现有的项目愿意接受(我一直很平庸)。那么自己的项目里就不会有这样的情况了。
第二点也许是年轻的冲劲。

我想写个演示程序,能比PowerPoint的特性高级(可能你觉得更糟)。当时,没有OpenOffice,只能选用
LaTex或者Magicpoint。而这两者更多的为文本展示做定制,展示旋转效果就不行了。我想写跨平台的,那时觉得Java
会是最好的选择。目标是完成一个用Java写的展示程序,它支持所有的旋转特效。我构想了一下,然后开始项目。

Infrastructure-wise everything was there: there was a mailing list, there was a
website, there was source code control (CVS). But there was no actual code for
people to work on. The only things I had were some ideas of what I wanted to do,
an itch to scratch and the right buzzwords. I actually wanted people to join in
creating this program and make it a truly collaborative project.

I started making designs (with some newly acquired UML knowledge) and sent them
around. Nothing happened. I tried to get people involved, but collaboratively
working on a design is very hard (besides, it is probably not the best way to
create software in the first place). After a while I gave up and the project
silently died, without ever producing a single line of code. Every month I was
reminded by the mailing list software that the project once existed, so I asked
it to be taken offline.

I learned a very valuable, but somewhat painful, lesson: when you announce
something and when you want people to get involved in your project, at least
make sure there is some code available. It does not have to be all finished, it
is OK if it is rough (in the beginning that is), but at least show that there is
a basis for people to work with and improve upon. Otherwise your project will go
where many many projects, including my own failed project, have gone: into
oblivion.

I eventually found my niche to help advance FLOSS, by making sure that the legal
underpinnings of FLOSS are tight through the gpl-violations.org project. In
retrospect I have never used, nor missed, the whirly effects in presentation
programs and found them to be increasingly irritating and distracting too much
from the content. I am happily using LaTeX beamer and occasionally (and less
happily) OpenOffice.org/LibreOffice to make presentations.
